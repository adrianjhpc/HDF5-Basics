%% Generated by Sphinx.
\def\sphinxdocclass{report}
\documentclass[letterpaper,10pt,english]{sphinxmanual}
\ifdefined\pdfpxdimen
   \let\sphinxpxdimen\pdfpxdimen\else\newdimen\sphinxpxdimen
\fi \sphinxpxdimen=.75bp\relax
\ifdefined\pdfimageresolution
    \pdfimageresolution= \numexpr \dimexpr1in\relax/\sphinxpxdimen\relax
\fi
%% let collapsible pdf bookmarks panel have high depth per default
\PassOptionsToPackage{bookmarksdepth=5}{hyperref}

\PassOptionsToPackage{booktabs}{sphinx}
\PassOptionsToPackage{colorrows}{sphinx}

\PassOptionsToPackage{warn}{textcomp}
\usepackage[utf8]{inputenc}
\ifdefined\DeclareUnicodeCharacter
% support both utf8 and utf8x syntaxes
  \ifdefined\DeclareUnicodeCharacterAsOptional
    \def\sphinxDUC#1{\DeclareUnicodeCharacter{"#1}}
  \else
    \let\sphinxDUC\DeclareUnicodeCharacter
  \fi
  \sphinxDUC{00A0}{\nobreakspace}
  \sphinxDUC{2500}{\sphinxunichar{2500}}
  \sphinxDUC{2502}{\sphinxunichar{2502}}
  \sphinxDUC{2514}{\sphinxunichar{2514}}
  \sphinxDUC{251C}{\sphinxunichar{251C}}
  \sphinxDUC{2572}{\textbackslash}
\fi
\usepackage{cmap}
\usepackage[T1]{fontenc}
\usepackage{amsmath,amssymb,amstext}
\usepackage{babel}



\usepackage{tgtermes}
\usepackage{tgheros}
\renewcommand{\ttdefault}{txtt}



\usepackage[Bjarne]{fncychap}
\usepackage{sphinx}

\fvset{fontsize=auto}
\usepackage{geometry}


% Include hyperref last.
\usepackage{hyperref}
% Fix anchor placement for figures with captions.
\usepackage{hypcap}% it must be loaded after hyperref.
% Set up styles of URL: it should be placed after hyperref.
\urlstyle{same}


\usepackage{sphinxmessages}




\title{HDF5 Tutorial Documentation}
\date{Sep 26, 2025}
\release{1.0}
\author{Adrian Jackson}
\newcommand{\sphinxlogo}{\vbox{}}
\renewcommand{\releasename}{Release}
\makeindex
\begin{document}

\ifdefined\shorthandoff
  \ifnum\catcode`\=\string=\active\shorthandoff{=}\fi
  \ifnum\catcode`\"=\active\shorthandoff{"}\fi
\fi

\pagestyle{empty}
\sphinxmaketitle
\pagestyle{plain}
\sphinxtableofcontents
\pagestyle{normal}
\phantomsection\label{\detokenize{index::doc}}


\sphinxAtStartPar
This tutorial is an introduction to HDF5. It shows how to read existing HDF5 files, and how to create and modify your own files.


\chapter{Compiling the examples}
\label{\detokenize{index:compiling-the-examples}}
\sphinxAtStartPar
We will use the ARCHER2 HPC system for this practical. If you do not have access to Cirrus please get in touch with the course organiser or lecturer/demonstrators for this practical.
Before compiling the examples make sure the HDF5 and Intel compiler modules are loaded:

\begin{sphinxVerbatim}[commandchars=\\\{\}]
\PYG{n}{module} \PYG{n}{swap} \PYG{n}{PrgEnv}\PYG{o}{\PYGZhy{}}\PYG{n}{cray} \PYG{n}{PrgEnv}\PYG{o}{\PYGZhy{}}\PYG{n}{gnu}
\PYG{n}{module} \PYG{n}{load} \PYG{n}{cray}\PYG{o}{\PYGZhy{}}\PYG{n}{hdf5}\PYG{o}{\PYGZhy{}}\PYG{n}{parallel}
\end{sphinxVerbatim}
\begin{description}
\sphinxlineitem{You can compile your HDF5 programs as shown below on ARCHER2 (note, we have not created a file called my\_first\_hdf5\_test.c so the command below will fail if you try it at the moment):}
\sphinxAtStartPar
cc my\_first\_hdf5\_test.c \sphinxhyphen{}o my\_first\_hdf5\_test

\sphinxlineitem{And run it on the login node like this:}
\sphinxAtStartPar
./my\_first\_hdf5\_test

\end{description}


\chapter{Viewing the contents of an HDF5 file}
\label{\detokenize{index:viewing-the-contents-of-an-hdf5-file}}
\sphinxAtStartPar
The commandline utility ‘h5dump’ allows to view the contents of a data file, like this:

\begin{sphinxVerbatim}[commandchars=\\\{\}]
\PYG{n}{h5dump} \PYG{n}{SampleFile}\PYG{o}{.}\PYG{n}{h5}
\end{sphinxVerbatim}

\sphinxAtStartPar
For large datasets only the header data can be displayed:

\begin{sphinxVerbatim}[commandchars=\\\{\}]
\PYG{n}{h5dump} \PYG{o}{\PYGZhy{}}\PYG{n}{H} \PYG{n}{SampleFile}\PYG{o}{.}\PYG{n}{h5}
\end{sphinxVerbatim}


\section{Reading an existing HDF5 file}
\label{\detokenize{index:reading-an-existing-hdf5-file}}
\sphinxAtStartPar
In this section we’re going to read data from an existing HDF5 file.


\chapter{View the contents}
\label{\detokenize{index:view-the-contents}}
\sphinxAtStartPar
An HDF5 file has a container or tree structure, very similar to folders or directories in a Linux or Windows file system. The root container is always called “/”, just as in a file system, and containers may contain other containers.

\sphinxAtStartPar
The leaves of the tree are datasets. A dataset has a header and a data array. The header contains information such as the name of the dataset, the dimensions of the data array, the type of its elements, other annotations and user\sphinxhyphen{}defined attributes.

\sphinxAtStartPar
To start with, download \sphinxcode{\sphinxupquote{example.h5}} to the login node:

\begin{sphinxVerbatim}[commandchars=\\\{\}]
\PYG{g+gp}{\PYGZdl{} }wget\PYG{+w}{ }https://adrianjhpc.github.io/HDF5\PYGZhy{}Basics/c/html/\PYGZus{}downloads/example.h5
\end{sphinxVerbatim}

\sphinxAtStartPar
We can have a look at its contents using ‘h5dump’:

\begin{sphinxVerbatim}[commandchars=\\\{\}]
\PYG{g+gp}{\PYGZdl{} }h5dump\PYG{+w}{ }example.h5
\PYG{g+go}{HDF5 \PYGZdq{}example.h5\PYGZdq{} \PYGZob{}}
\PYG{g+go}{GROUP \PYGZdq{}/\PYGZdq{} \PYGZob{}}
\PYG{g+go}{   DATASET \PYGZdq{}dset\PYGZdq{} \PYGZob{}}
\PYG{g+go}{      DATATYPE  H5T\PYGZus{}STD\PYGZus{}I32LE}
\PYG{g+go}{      DATASPACE  SIMPLE \PYGZob{} ( 6, 15 ) / ( 6, 15 ) \PYGZcb{}}
\PYG{g+go}{      DATA \PYGZob{}}
\PYG{g+go}{      (0,0): 1, 2, 3, 4, 5, 6, 7, 8, 9, 10, 11, 12, 13, 14, 15,}
\PYG{g+go}{      (1,0): 16, 17, 18, 19, 20, 21, 22, 23, 24, 25, 26, 27, 28, 29, 30,}
\PYG{g+go}{      (2,0): 31, 32, 33, 34, 35, 36, 37, 38, 39, 40, 41, 42, 43, 44, 45,}
\PYG{g+go}{      (3,0): 46, 47, 48, 49, 50, 51, 52, 53, 54, 55, 56, 57, 58, 59, 60,}
\PYG{g+go}{      (4,0): 61, 62, 63, 64, 65, 66, 67, 68, 69, 70, 71, 72, 73, 74, 75,}
\PYG{g+go}{      (5,0): 76, 77, 78, 79, 80, 81, 82, 83, 84, 85, 86, 87, 88, 89, 90}
\PYG{g+go}{      \PYGZcb{}}
\PYG{g+go}{   \PYGZcb{}}
\PYG{g+go}{\PYGZcb{}}
\PYG{g+go}{\PYGZcb{}}
\end{sphinxVerbatim}

\sphinxAtStartPar
This file has one dataset which is located in the root container (group “/”) of the file.
The dataset is called “dset” and contains a ‘SIMPLE’ dataspace which is an array \sphinxhyphen{} in this case a 2\sphinxhyphen{}dimensional array of size (6, 15).
The type of the elements in the array is ‘H5T\_STD\_I32BE’, i.e. the values are 32\sphinxhyphen{}bit big\sphinxhyphen{}endian integers.


\chapter{Accessing the file}
\label{\detokenize{index:accessing-the-file}}
\sphinxAtStartPar
Now we’re going to write some C code to open the data file and close it again:

\begin{sphinxVerbatim}[commandchars=\\\{\}]
\PYG{c+c1}{\PYGZsh{}include \PYGZdq{}hdf5.h\PYGZdq{}}

\PYG{n+nb}{int} \PYG{n}{main}\PYG{p}{(}\PYG{p}{)} \PYG{p}{\PYGZob{}}

     \PYG{n}{hid\PYGZus{}t} \PYG{n}{file\PYGZus{}id}\PYG{p}{;}
     \PYG{n}{herr\PYGZus{}t} \PYG{n}{status}\PYG{p}{;}

     \PYG{n}{file\PYGZus{}id} \PYG{o}{=} \PYG{n}{H5Fopen}\PYG{p}{(}\PYG{l+s+s2}{\PYGZdq{}}\PYG{l+s+s2}{example.h5}\PYG{l+s+s2}{\PYGZdq{}}\PYG{p}{,} \PYG{n}{H5F\PYGZus{}ACC\PYGZus{}RDWR}\PYG{p}{,} \PYG{n}{H5P\PYGZus{}DEFAULT}\PYG{p}{)}\PYG{p}{;}
     \PYG{n}{printf}\PYG{p}{(}\PYG{l+s+s2}{\PYGZdq{}}\PYG{l+s+s2}{Opened file \PYGZhy{} id: }\PYG{l+s+si}{\PYGZpc{}i}\PYG{l+s+se}{\PYGZbs{}n}\PYG{l+s+s2}{\PYGZdq{}}\PYG{p}{,} \PYG{n}{file\PYGZus{}id}\PYG{p}{)}\PYG{p}{;}

     \PYG{n}{status} \PYG{o}{=} \PYG{n}{H5Fclose}\PYG{p}{(}\PYG{n}{file\PYGZus{}id}\PYG{p}{)}\PYG{p}{;}

     \PYG{k}{return} \PYG{l+m+mi}{0}\PYG{p}{;}
\PYG{p}{\PYGZcb{}}
\end{sphinxVerbatim}

\sphinxAtStartPar
This opens the file ‘example.h5’ and prints out a message. The options for creating or opening a file are:
\begin{description}
\sphinxlineitem{\sphinxstylestrong{H5F\_ACC\_EXCL}:}
\sphinxAtStartPar
creates a new file and fails if the file already exists. This is the default.

\sphinxlineitem{\sphinxstylestrong{H5F\_ACC\_TRUNC}:}
\sphinxAtStartPar
creates a new file or opens and overwrites an existing one

\sphinxlineitem{\sphinxstylestrong{H5F\_ACC\_RDONLY}:}
\sphinxAtStartPar
opens an existing file with read\sphinxhyphen{}only access

\sphinxlineitem{\sphinxstylestrong{H5F\_ACC\_RDWR}:}
\sphinxAtStartPar
opens an existing file for reading and writing

\end{description}

\sphinxAtStartPar
See \sphinxurl{http://www.hdfgroup.org/HDF5/doc/UG/08\_TheFile.html} for more details.

\sphinxAtStartPar
To run this example, create a file (for example ‘tutorial.c’) with the C source code above and compile it as explained in the section ‘Before you start’.


\chapter{Reading data}
\label{\detokenize{index:reading-data}}
\sphinxAtStartPar
Now modify the above example to read the data. As we have seen from the output of h5dump, there is one dataset called “dset” in the root group “/”. First, we open the dataset whose contents we’re going to read (this assumes that the file ‘file\_id’ has been opened already):

\begin{sphinxVerbatim}[commandchars=\\\{\}]
\PYG{n}{dataset\PYGZus{}id} \PYG{o}{=} \PYG{n}{H5Dopen2}\PYG{p}{(}\PYG{n}{file\PYGZus{}id}\PYG{p}{,} \PYG{l+s+s2}{\PYGZdq{}}\PYG{l+s+s2}{/dset}\PYG{l+s+s2}{\PYGZdq{}}\PYG{p}{,} \PYG{n}{H5P\PYGZus{}DEFAULT}\PYG{p}{)}\PYG{p}{;}
\end{sphinxVerbatim}

\sphinxAtStartPar
Then create an array that is large enough to hold the dataset from the file:

\begin{sphinxVerbatim}[commandchars=\\\{\}]
\PYG{n+nb}{int} \PYG{n}{dset\PYGZus{}data}\PYG{p}{[}\PYG{l+m+mi}{6}\PYG{p}{]}\PYG{p}{[}\PYG{l+m+mi}{15}\PYG{p}{]}\PYG{p}{;}
\PYG{n}{status} \PYG{o}{=} \PYG{n}{H5Dread}\PYG{p}{(}\PYG{n}{dataset\PYGZus{}id}\PYG{p}{,} \PYG{n}{H5T\PYGZus{}NATIVE\PYGZus{}INT}\PYG{p}{,} \PYG{n}{H5S\PYGZus{}ALL}\PYG{p}{,} \PYG{n}{H5S\PYGZus{}ALL}\PYG{p}{,} \PYG{n}{H5P\PYGZus{}DEFAULT}\PYG{p}{,} \PYG{n}{dset\PYGZus{}data}\PYG{p}{)}\PYG{p}{;}
\end{sphinxVerbatim}

\sphinxAtStartPar
Once the dataset contents have been read you can print them out to the commandline.


\section{Modify an HDF5 file}
\label{\detokenize{index:modify-an-hdf5-file}}
\sphinxAtStartPar
This section shows how to modify the structure of an HDF5 file, how to create and write datasets, and how to attach
attributes to groups or datasets.


\chapter{Creating groups}
\label{\detokenize{index:creating-groups}}
\sphinxAtStartPar
The example we’ve been reading has only one group, the root container “/”. Now we’re going to to create a few more groups.

\sphinxAtStartPar
After opening the file and reading its contents, create a new group:

\begin{sphinxVerbatim}[commandchars=\\\{\}]
\PYG{n}{group\PYGZus{}id} \PYG{o}{=} \PYG{n}{H5Gcreate}\PYG{p}{(}\PYG{n}{file\PYGZus{}id}\PYG{p}{,} \PYG{l+s+s2}{\PYGZdq{}}\PYG{l+s+s2}{/Earthquake}\PYG{l+s+s2}{\PYGZdq{}}\PYG{p}{,} \PYG{n}{H5P\PYGZus{}DEFAULT}\PYG{p}{,} \PYG{n}{H5P\PYGZus{}DEFAULT}\PYG{p}{,} \PYG{n}{H5P\PYGZus{}DEFAULT}\PYG{p}{)}\PYG{p}{;}
\end{sphinxVerbatim}

\sphinxAtStartPar
Note that groups can be created using absolute paths or relative paths:

\begin{sphinxVerbatim}[commandchars=\\\{\}]
\PYG{n}{l\PYGZus{}id} \PYG{o}{=} \PYG{n}{H5Gcreate}\PYG{p}{(}\PYG{n}{file\PYGZus{}id}\PYG{p}{,} \PYG{l+s+s2}{\PYGZdq{}}\PYG{l+s+s2}{/Earthquake/Laquila}\PYG{l+s+s2}{\PYGZdq{}}\PYG{p}{,} \PYG{n}{H5P\PYGZus{}DEFAULT}\PYG{p}{,} \PYG{n}{H5P\PYGZus{}DEFAULT}\PYG{p}{,} \PYG{n}{H5P\PYGZus{}DEFAULT}\PYG{p}{)}\PYG{p}{;}
\end{sphinxVerbatim}

\sphinxAtStartPar
This is the same as:

\begin{sphinxVerbatim}[commandchars=\\\{\}]
\PYG{n}{l\PYGZus{}id} \PYG{o}{=} \PYG{n}{H5Gcreate}\PYG{p}{(}\PYG{n}{group\PYGZus{}id}\PYG{p}{,} \PYG{l+s+s2}{\PYGZdq{}}\PYG{l+s+s2}{Laquila}\PYG{l+s+s2}{\PYGZdq{}}\PYG{p}{,} \PYG{n}{H5P\PYGZus{}DEFAULT}\PYG{p}{,} \PYG{n}{H5P\PYGZus{}DEFAULT}\PYG{p}{,} \PYG{n}{H5P\PYGZus{}DEFAULT}\PYG{p}{)}\PYG{p}{;}
\end{sphinxVerbatim}

\sphinxAtStartPar
Now create the following group structure:

\noindent\sphinxincludegraphics{{group_structure}.png}


\chapter{Creating a new dataset}
\label{\detokenize{index:creating-a-new-dataset}}
\sphinxAtStartPar
Now we’re going to add a new dataset in the group “/Earthquake/Laquila/Traces”. A dataset has a name and is characterised by the dataspace (the shape of the array) and the datatype (the layout of the stored elements). For our scenario we are going to create a dataset that contains a 2\sphinxhyphen{}dimensional array of integers.

\sphinxAtStartPar
First we create the dataspace, which is a 2\sphinxhyphen{}dimensional array in our scenario. For example to create an array with dimensions (5, 10):

\begin{sphinxVerbatim}[commandchars=\\\{\}]
\PYG{n}{hsize\PYGZus{}t} \PYG{n}{dims}\PYG{p}{[}\PYG{l+m+mi}{2}\PYG{p}{]}\PYG{p}{;}
\PYG{n}{dims}\PYG{p}{[}\PYG{l+m+mi}{0}\PYG{p}{]} \PYG{o}{=} \PYG{l+m+mi}{5}\PYG{p}{;}
\PYG{n}{dims}\PYG{p}{[}\PYG{l+m+mi}{1}\PYG{p}{]} \PYG{o}{=} \PYG{l+m+mi}{10}\PYG{p}{;}
\PYG{n}{dataspace\PYGZus{}id} \PYG{o}{=} \PYG{n}{H5Screate\PYGZus{}simple}\PYG{p}{(}\PYG{l+m+mi}{2}\PYG{p}{,} \PYG{n}{dims}\PYG{p}{,} \PYG{n}{NULL}\PYG{p}{)}\PYG{p}{;}
\end{sphinxVerbatim}

\sphinxAtStartPar
A SIMPLE dataspace represents a multidimensional array. There are also SCALAR dataspaces (containing just one element) and NULL dataspaces that contain no elements. See \sphinxurl{http://www.hdfgroup.org/HDF5/doc/UG/UG\_frame12Dataspaces.html} for more details.

\sphinxAtStartPar
Now we can use the new dataspace to create the dataset “day1” within the group “Laquila”:

\begin{sphinxVerbatim}[commandchars=\\\{\}]
\PYG{n}{dataset\PYGZus{}id} \PYG{o}{=} \PYG{n}{H5Dcreate2}\PYG{p}{(}\PYG{n}{l\PYGZus{}id}\PYG{p}{,} \PYG{l+s+s2}{\PYGZdq{}}\PYG{l+s+s2}{day1}\PYG{l+s+s2}{\PYGZdq{}}\PYG{p}{,} \PYG{n}{H5T\PYGZus{}NATIVE\PYGZus{}INT}\PYG{p}{,} \PYG{n}{dataspace\PYGZus{}id}\PYG{p}{,} \PYG{n}{H5P\PYGZus{}DEFAULT}\PYG{p}{,} \PYG{n}{H5P\PYGZus{}DEFAULT}\PYG{p}{,} \PYG{n}{H5P\PYGZus{}DEFAULT}\PYG{p}{)}\PYG{p}{;}
\end{sphinxVerbatim}

\sphinxAtStartPar
There are many predefined datatypes. The native type H5T\_NATIVE\_INT corresponds to a C int type. For example, on an Intel based PC, this type is the same as H5T\_STD\_I32LE. See \sphinxurl{http://www.hdfgroup.org/HDF5/doc/UG/UG\_frame11Datatypes.html} for more details.


\chapter{Writing to a dataset}
\label{\detokenize{index:writing-to-a-dataset}}
\sphinxAtStartPar
Now create an integer array dset\_data, with the same dimensions as the dataspace (5, 10), and fill it with some data. Then write its contents to the dataset:

\begin{sphinxVerbatim}[commandchars=\\\{\}]
\PYG{n+nb}{int} \PYG{n}{dset\PYGZus{}data}\PYG{p}{[}\PYG{l+m+mi}{5}\PYG{p}{]}\PYG{p}{[}\PYG{l+m+mi}{10}\PYG{p}{]}\PYG{p}{;}
\PYG{n}{dset\PYGZus{}data}\PYG{p}{[}\PYG{l+m+mi}{0}\PYG{p}{]}\PYG{p}{[}\PYG{l+m+mi}{0}\PYG{p}{]} \PYG{o}{=} \PYG{l+m+mi}{23}\PYG{p}{;}
\PYG{o}{/}\PYG{o}{*} \PYG{n}{add} \PYG{n}{more} \PYG{n}{data} \PYG{p}{(}\PYG{n}{integer} \PYG{n}{values}\PYG{p}{)} \PYG{n}{to} \PYG{n}{the} \PYG{n}{array} \PYG{n}{here} \PYG{o}{*}\PYG{o}{/}
\PYG{o}{.}\PYG{o}{.}\PYG{o}{.}
\PYG{n}{status} \PYG{o}{=} \PYG{n}{H5Dwrite}\PYG{p}{(}\PYG{n}{dataset\PYGZus{}id}\PYG{p}{,} \PYG{n}{H5T\PYGZus{}NATIVE\PYGZus{}INT}\PYG{p}{,} \PYG{n}{H5S\PYGZus{}ALL}\PYG{p}{,} \PYG{n}{H5S\PYGZus{}ALL}\PYG{p}{,} \PYG{n}{H5P\PYGZus{}DEFAULT}\PYG{p}{,} \PYG{n}{dset\PYGZus{}data}\PYG{p}{)}\PYG{p}{;}
\end{sphinxVerbatim}

\sphinxAtStartPar
Don’t forget to close the dataset when finished:

\begin{sphinxVerbatim}[commandchars=\\\{\}]
\PYG{n}{status} \PYG{o}{=} \PYG{n}{H5Dclose}\PYG{p}{(}\PYG{n}{dataset\PYGZus{}id}\PYG{p}{)}\PYG{p}{;}
\end{sphinxVerbatim}

\sphinxAtStartPar
Try to create more datasets in various groups and write to and read from them. You can always check the contents of your HDF5 file using h5dump.


\chapter{Attributes}
\label{\detokenize{index:attributes}}
\sphinxAtStartPar
Attributes can be attached to HDF5 datasets or groups. An attribute has two parts: a name and a value. See \sphinxurl{http://www.hdfgroup.org/HDF5/doc/UG/UG\_frame13Attributes.html} for more information. Attributes are defined with a dataspace and type in the same way as datasets.

\sphinxAtStartPar
Let’s create a string attribute for the root group of our HDF5 file, stating the author:

\begin{sphinxVerbatim}[commandchars=\\\{\}]
\PYG{n}{char} \PYG{n}{value}\PYG{p}{[}\PYG{p}{]} \PYG{o}{=} \PYG{l+s+s2}{\PYGZdq{}}\PYG{l+s+s2}{Adrian Jackson}\PYG{l+s+s2}{\PYGZdq{}}\PYG{p}{;}
\PYG{n}{len\PYGZus{}value} \PYG{o}{=} \PYG{n}{strlen}\PYG{p}{(}\PYG{n}{value}\PYG{p}{)}\PYG{o}{+}\PYG{l+m+mi}{1}\PYG{p}{;}
\PYG{n}{attr\PYGZus{}id}  \PYG{o}{=} \PYG{n}{H5Screate}\PYG{p}{(}\PYG{n}{H5S\PYGZus{}SCALAR}\PYG{p}{)}\PYG{p}{;}
\PYG{n}{attr\PYGZus{}type} \PYG{o}{=} \PYG{n}{H5Tcopy}\PYG{p}{(}\PYG{n}{H5T\PYGZus{}C\PYGZus{}S1}\PYG{p}{)}\PYG{p}{;}
\PYG{n}{H5Tset\PYGZus{}size}\PYG{p}{(}\PYG{n}{attr\PYGZus{}type}\PYG{p}{,} \PYG{n}{len\PYGZus{}value}\PYG{p}{)}\PYG{p}{;}
\PYG{n}{H5Tset\PYGZus{}strpad}\PYG{p}{(}\PYG{n}{attr\PYGZus{}type}\PYG{p}{,} \PYG{n}{H5T\PYGZus{}STR\PYGZus{}NULLTERM}\PYG{p}{)}\PYG{p}{;}
\PYG{n}{attr} \PYG{o}{=} \PYG{n}{H5Acreate2}\PYG{p}{(}\PYG{n}{file\PYGZus{}id}\PYG{p}{,} \PYG{l+s+s2}{\PYGZdq{}}\PYG{l+s+s2}{author}\PYG{l+s+s2}{\PYGZdq{}}\PYG{p}{,} \PYG{n}{attr\PYGZus{}type}\PYG{p}{,} \PYG{n}{attr\PYGZus{}id}\PYG{p}{,} \PYG{n}{H5P\PYGZus{}DEFAULT}\PYG{p}{,} \PYG{n}{H5P\PYGZus{}DEFAULT}\PYG{p}{)}\PYG{p}{;}
\PYG{n}{status} \PYG{o}{=} \PYG{n}{H5Awrite}\PYG{p}{(}\PYG{n}{attr}\PYG{p}{,} \PYG{n}{attr\PYGZus{}type}\PYG{p}{,} \PYG{n}{value}\PYG{p}{)}\PYG{p}{;}
\end{sphinxVerbatim}

\sphinxAtStartPar
The attribute is named ‘author’ and has a scalar dataspace (one element) of type C string. The size is the number of characters in the attribute value (10) plus one for the null terminator.

\sphinxAtStartPar
Now add an attribute to the dataset that you created above, within group ‘Laquila’, using the same technique, for various types. For example:
\begin{itemize}
\item {} 
\sphinxAtStartPar
Integer: H5T\_NATIVE\_INT

\item {} 
\sphinxAtStartPar
Float: H5T\_NATIVE\_FLOAT

\item {} 
\sphinxAtStartPar
Double: H5T\_NATIVE\_DOUBLE

\end{itemize}

\sphinxAtStartPar
Remember to use dataset\_id instead of file\_id if you create an attribute for a dataset, or group\_id if you’re attaching an attribute to a group.


\section{Modifying the HDF5 file structure}
\label{\detokenize{index:modifying-the-hdf5-file-structure}}
\sphinxAtStartPar
An HDF5 file is structured just like a file system, with directories or folders (called containers) and files (called datasets).
The library allows to modify this structure in the same way as you can modify a file system.


\chapter{Moving a dataset}
\label{\detokenize{index:moving-a-dataset}}
\sphinxAtStartPar
You can easily move the dataset “dset” from the root container into the container “/Earthquake/Laquila/Traces/”,
first opening both groups and then moving the dataset from one to the other.
The following also renames the dataset from “dset” to “day2”:

\begin{sphinxVerbatim}[commandchars=\\\{\}]
\PYG{n}{file\PYGZus{}id} \PYG{o}{=} \PYG{n}{H5Fopen}\PYG{p}{(}\PYG{l+s+s2}{\PYGZdq{}}\PYG{l+s+s2}{example.h5}\PYG{l+s+s2}{\PYGZdq{}}\PYG{p}{,} \PYG{n}{H5F\PYGZus{}ACC\PYGZus{}RDWR}\PYG{p}{,} \PYG{n}{H5P\PYGZus{}DEFAULT}\PYG{p}{)}\PYG{p}{;}
\PYG{n}{group\PYGZus{}id} \PYG{o}{=} \PYG{n}{H5Gopen}\PYG{p}{(}\PYG{n}{file\PYGZus{}id}\PYG{p}{,} \PYG{l+s+s2}{\PYGZdq{}}\PYG{l+s+s2}{/Earthquake/Laquila/Traces}\PYG{l+s+s2}{\PYGZdq{}}\PYG{p}{,} \PYG{n}{H5P\PYGZus{}DEFAULT}\PYG{p}{)}\PYG{p}{;}

\PYG{n}{H5Lmove}\PYG{p}{(}\PYG{n}{file\PYGZus{}id}\PYG{p}{,} \PYG{l+s+s2}{\PYGZdq{}}\PYG{l+s+s2}{dset}\PYG{l+s+s2}{\PYGZdq{}}\PYG{p}{,} \PYG{n}{group\PYGZus{}id}\PYG{p}{,} \PYG{l+s+s2}{\PYGZdq{}}\PYG{l+s+s2}{day2}\PYG{l+s+s2}{\PYGZdq{}}\PYG{p}{,} \PYG{n}{H5P\PYGZus{}DEFAULT}\PYG{p}{,} \PYG{n}{H5P\PYGZus{}DEFAULT}\PYG{p}{)}\PYG{p}{;}
\end{sphinxVerbatim}


\chapter{Symbolic links}
\label{\detokenize{index:symbolic-links}}
\sphinxAtStartPar
It is also possible to create symbolic links to point to objects in other locations in the HDF5 file structure.
Linked objects can be groups or datasets.
For example, create a soft link to the dataset created above from within another group:

\begin{sphinxVerbatim}[commandchars=\\\{\}]
\PYG{n}{H5Lcreate\PYGZus{}soft}\PYG{p}{(}\PYG{o}{\PYGZlt{}}\PYG{n}{source\PYGZus{}name}\PYG{o}{\PYGZgt{}}\PYG{p}{,} \PYG{n}{group\PYGZus{}id}\PYG{p}{,} \PYG{o}{\PYGZlt{}}\PYG{n}{target\PYGZus{}name}\PYG{o}{\PYGZgt{}}\PYG{p}{,} \PYG{n}{H5P\PYGZus{}DEFAULT}\PYG{p}{,} \PYG{n}{H5P\PYGZus{}DEFAULT}\PYG{p}{)}\PYG{p}{;}
\end{sphinxVerbatim}

\sphinxAtStartPar
The source name is either an absolute path of the source of the link, or it a relative path within group \sphinxtitleref{group\_id}.
The target is resolved at runtime and is a name of an object in the group \sphinxtitleref{group\_id}.

\sphinxAtStartPar
The link command is very similar to moving files above, but note that in the command for creating a soft link,
the source and target names can’t be relative paths to different groups.


\chapter{External links}
\label{\detokenize{index:external-links}}
\sphinxAtStartPar
External links are links from an HDF5 file to an object in another HDF5 file.
Once created the external object behaves like it is part of the file.

\sphinxAtStartPar
Download the dataset \sphinxcode{\sphinxupquote{NapaValley.h5}}.
Then link a group ‘Earthquake/NapaValley/’ in your file to the group ‘Traces’ in the external file:

\begin{sphinxVerbatim}[commandchars=\\\{\}]
\PYG{n}{H5Lcreate\PYGZus{}external}\PYG{p}{(}\PYG{l+s+s2}{\PYGZdq{}}\PYG{l+s+s2}{NapaValley.h5}\PYG{l+s+s2}{\PYGZdq{}}\PYG{p}{,} \PYG{o}{\PYGZlt{}}\PYG{n}{TARGET\PYGZus{}GROUP}\PYG{o}{\PYGZgt{}}\PYG{p}{,} \PYG{n}{file\PYGZus{}id}\PYG{p}{,} \PYG{o}{\PYGZlt{}}\PYG{n}{SOURCE\PYGZus{}GROUP}\PYG{o}{\PYGZgt{}}\PYG{p}{,} \PYG{n}{H5P\PYGZus{}DEFAULT}\PYG{p}{,} \PYG{n}{H5P\PYGZus{}DEFAULT}\PYG{p}{)}\PYG{p}{;}
\end{sphinxVerbatim}

\sphinxAtStartPar
In the command above replace \sphinxtitleref{TARGET\_GROUP} with the group in the external file and \sphinxtitleref{SOURCE\_GROUP}
with a new group in your file that points to the external group.
Now you can read this new group as if it was part of the source HDF5 file.


\section{Partial I/O}
\label{\detokenize{index:partial-i-o}}

\chapter{Regions and hyperslabs}
\label{\detokenize{index:regions-and-hyperslabs}}
\sphinxAtStartPar
As HDF5 is commonly used when writing or reading files in a parallel application,
it is possible to select certain elements of a dataset rather than the whole array,
thus allowing to write different portions of a file or dataset from each process.
See \sphinxurl{http://www.hdfgroup.org/HDF5/doc/UG/12\_Dataspaces.html\#DTransfer} for more information.
Regions of a dataset are called hyperslabs.

\noindent\sphinxincludegraphics{{hyperslab2}.png}

\sphinxAtStartPar
For example you would use this when writing an MPI application in which data is distributed across processes.
As shown below each row (or column) of a shared array is read by a different process
and each process calculates a result from this data and writes it to a shared output file.
The selection of hyperslabs provides you with a view of the dataset region that each process reads or writes,
without having to worry about the physical location in the file or its shape and size.
The HDF5 library also supports the selection of independent elements of a dataset
and creating unions of selections.

\sphinxAtStartPar
An HDF5 hyperslab is defined by the parameters:
\begin{itemize}
\item {} 
\sphinxAtStartPar
offset

\item {} 
\sphinxAtStartPar
stride

\item {} 
\sphinxAtStartPar
count (the number of blocks)

\item {} 
\sphinxAtStartPar
block size

\end{itemize}

\noindent\sphinxincludegraphics{{hyperslab3}.png}


\chapter{Selecting a hyperslab}
\label{\detokenize{index:selecting-a-hyperslab}}
\sphinxAtStartPar
In the following example, you’re going to select and modify a hyperslab of the dataset you created above.

\sphinxAtStartPar
First create a dataspace of the same dimensions as the target dataset:

\begin{sphinxVerbatim}[commandchars=\\\{\}]
\PYG{n}{hsize\PYGZus{}t} \PYG{n}{dims}\PYG{p}{[}\PYG{l+m+mi}{2}\PYG{p}{]} \PYG{o}{=} \PYG{p}{\PYGZob{}}\PYG{n}{DIM0}\PYG{p}{,} \PYG{n}{DIM1}\PYG{p}{\PYGZcb{}}\PYG{p}{;}
\PYG{n}{space} \PYG{o}{=} \PYG{n}{H5Screate\PYGZus{}simple} \PYG{p}{(}\PYG{l+m+mi}{2}\PYG{p}{,} \PYG{n}{dims}\PYG{p}{,} \PYG{n}{NULL}\PYG{p}{)}\PYG{p}{;}
\end{sphinxVerbatim}

\sphinxAtStartPar
and create a data array, for example:

\begin{sphinxVerbatim}[commandchars=\\\{\}]
\PYG{n+nb}{int} \PYG{n}{data}\PYG{p}{[}\PYG{n}{DIM0}\PYG{p}{]}\PYG{p}{[}\PYG{n}{DIM1}\PYG{p}{]}\PYG{p}{;}
\PYG{n+nb}{int} \PYG{n}{i}\PYG{p}{,}\PYG{n}{j}\PYG{p}{;}
\PYG{k}{for} \PYG{p}{(}\PYG{n}{i}\PYG{o}{=}\PYG{l+m+mi}{0}\PYG{p}{;} \PYG{n}{i}\PYG{o}{\PYGZlt{}}\PYG{n}{DIM0}\PYG{p}{;} \PYG{n}{i}\PYG{o}{+}\PYG{o}{+}\PYG{p}{)}
    \PYG{k}{for} \PYG{p}{(}\PYG{n}{j}\PYG{o}{=}\PYG{l+m+mi}{0}\PYG{p}{;} \PYG{n}{j}\PYG{o}{\PYGZlt{}}\PYG{n}{DIM1}\PYG{p}{;} \PYG{n}{j}\PYG{o}{+}\PYG{o}{+}\PYG{p}{)}
        \PYG{n}{data}\PYG{p}{[}\PYG{n}{i}\PYG{p}{]}\PYG{p}{[}\PYG{n}{j}\PYG{p}{]}\PYG{o}{=}\PYG{p}{(}\PYG{n}{i}\PYG{o}{+}\PYG{n}{j}\PYG{p}{)}\PYG{o}{*}\PYG{l+m+mi}{100}\PYG{p}{;}
\end{sphinxVerbatim}

\sphinxAtStartPar
Then select a region by defining the start and the number of points to write:

\begin{sphinxVerbatim}[commandchars=\\\{\}]
\PYG{n}{hsize\PYGZus{}t} \PYG{n}{start}\PYG{p}{[}\PYG{l+m+mi}{2}\PYG{p}{]}\PYG{p}{,} \PYG{n}{count}\PYG{p}{[}\PYG{l+m+mi}{2}\PYG{p}{]}\PYG{p}{,} \PYG{n}{stride}\PYG{p}{[}\PYG{l+m+mi}{2}\PYG{p}{]}\PYG{p}{,} \PYG{n}{block}\PYG{p}{[}\PYG{l+m+mi}{2}\PYG{p}{]}\PYG{p}{;}
\PYG{n}{start}\PYG{p}{[}\PYG{l+m+mi}{0}\PYG{p}{]} \PYG{o}{=} \PYG{l+m+mi}{1}\PYG{p}{;}
\PYG{n}{start}\PYG{p}{[}\PYG{l+m+mi}{1}\PYG{p}{]} \PYG{o}{=} \PYG{l+m+mi}{2}\PYG{p}{;}
\PYG{n}{count}\PYG{p}{[}\PYG{l+m+mi}{0}\PYG{p}{]} \PYG{o}{=} \PYG{l+m+mi}{2}\PYG{p}{;}
\PYG{n}{count}\PYG{p}{[}\PYG{l+m+mi}{1}\PYG{p}{]} \PYG{o}{=} \PYG{l+m+mi}{3}\PYG{p}{;}

\PYG{n}{status} \PYG{o}{=} \PYG{n}{H5Sselect\PYGZus{}hyperslab} \PYG{p}{(}\PYG{n}{space}\PYG{p}{,} \PYG{n}{H5S\PYGZus{}SELECT\PYGZus{}SET}\PYG{p}{,} \PYG{n}{start}\PYG{p}{,} \PYG{n}{NULL}\PYG{p}{,} \PYG{n}{count}\PYG{p}{,} \PYG{n}{NULL}\PYG{p}{)}\PYG{p}{;}
\end{sphinxVerbatim}

\sphinxAtStartPar
This selects the hyperslab (in this case a rectangle) of size (2,3) located at (1,2) in the array, like this:

\noindent\sphinxincludegraphics{{hyperslab1}.png}

\sphinxAtStartPar
Now write the data:

\begin{sphinxVerbatim}[commandchars=\\\{\}]
\PYG{n}{status} \PYG{o}{=} \PYG{n}{H5Dwrite} \PYG{p}{(}\PYG{n}{dataset\PYGZus{}id}\PYG{p}{,} \PYG{n}{H5T\PYGZus{}NATIVE\PYGZus{}INT}\PYG{p}{,} \PYG{n}{H5S\PYGZus{}ALL}\PYG{p}{,} \PYG{n}{space}\PYG{p}{,} \PYG{n}{H5P\PYGZus{}DEFAULT}\PYG{p}{,} \PYG{n}{data}\PYG{p}{)}\PYG{p}{;}
\end{sphinxVerbatim}

\sphinxAtStartPar
You can also change the size of blocks and the stride between the blocks, for example:

\begin{sphinxVerbatim}[commandchars=\\\{\}]
\PYG{n}{stride}\PYG{p}{[}\PYG{l+m+mi}{0}\PYG{p}{]} \PYG{o}{=} \PYG{l+m+mi}{3}\PYG{p}{;}
\PYG{n}{stride}\PYG{p}{[}\PYG{l+m+mi}{1}\PYG{p}{]} \PYG{o}{=} \PYG{l+m+mi}{3}\PYG{p}{;}

\PYG{n}{block}\PYG{p}{[}\PYG{l+m+mi}{0}\PYG{p}{]} \PYG{o}{=} \PYG{l+m+mi}{2}\PYG{p}{;}
\PYG{n}{block}\PYG{p}{[}\PYG{l+m+mi}{1}\PYG{p}{]} \PYG{o}{=} \PYG{l+m+mi}{2}\PYG{p}{;}

\PYG{n}{status} \PYG{o}{=} \PYG{n}{H5Sselect\PYGZus{}hyperslab} \PYG{p}{(}\PYG{n}{space}\PYG{p}{,} \PYG{n}{H5S\PYGZus{}SELECT\PYGZus{}SET}\PYG{p}{,} \PYG{n}{start}\PYG{p}{,} \PYG{n}{stride}\PYG{p}{,} \PYG{n}{count}\PYG{p}{,} \PYG{n}{block}\PYG{p}{)}\PYG{p}{;}
\end{sphinxVerbatim}

\sphinxAtStartPar
Use \sphinxtitleref{h5dump} to check how the dataset looks now. Which elements have been replaced by new ones?


\chapter{Selecting elements}
\label{\detokenize{index:selecting-elements}}
\sphinxAtStartPar
You can also select single elements from a dataset, for example to write a sequence of points:

\begin{sphinxVerbatim}[commandchars=\\\{\}]
\PYG{n}{coord}\PYG{p}{[}\PYG{l+m+mi}{0}\PYG{p}{]}\PYG{p}{[}\PYG{l+m+mi}{0}\PYG{p}{]} \PYG{o}{=} \PYG{l+m+mi}{0}\PYG{p}{;} \PYG{n}{coord}\PYG{p}{[}\PYG{l+m+mi}{0}\PYG{p}{]}\PYG{p}{[}\PYG{l+m+mi}{1}\PYG{p}{]} \PYG{o}{=} \PYG{l+m+mi}{0}\PYG{p}{;}
\PYG{n}{coord}\PYG{p}{[}\PYG{l+m+mi}{1}\PYG{p}{]}\PYG{p}{[}\PYG{l+m+mi}{0}\PYG{p}{]} \PYG{o}{=} \PYG{l+m+mi}{3}\PYG{p}{;} \PYG{n}{coord}\PYG{p}{[}\PYG{l+m+mi}{1}\PYG{p}{]}\PYG{p}{[}\PYG{l+m+mi}{1}\PYG{p}{]} \PYG{o}{=} \PYG{l+m+mi}{3}\PYG{p}{;}
\PYG{n}{coord}\PYG{p}{[}\PYG{l+m+mi}{2}\PYG{p}{]}\PYG{p}{[}\PYG{l+m+mi}{0}\PYG{p}{]} \PYG{o}{=} \PYG{l+m+mi}{3}\PYG{p}{;} \PYG{n}{coord}\PYG{p}{[}\PYG{l+m+mi}{2}\PYG{p}{]}\PYG{p}{[}\PYG{l+m+mi}{1}\PYG{p}{]} \PYG{o}{=} \PYG{l+m+mi}{5}\PYG{p}{;}
\PYG{n}{coord}\PYG{p}{[}\PYG{l+m+mi}{3}\PYG{p}{]}\PYG{p}{[}\PYG{l+m+mi}{0}\PYG{p}{]} \PYG{o}{=} \PYG{l+m+mi}{5}\PYG{p}{;} \PYG{n}{coord}\PYG{p}{[}\PYG{l+m+mi}{3}\PYG{p}{]}\PYG{p}{[}\PYG{l+m+mi}{1}\PYG{p}{]} \PYG{o}{=} \PYG{l+m+mi}{6}\PYG{p}{;}

\PYG{n}{status} \PYG{o}{=} \PYG{n}{H5Sselect\PYGZus{}elements}\PYG{p}{(}\PYG{n}{file\PYGZus{}id}\PYG{p}{,} \PYG{n}{H5S\PYGZus{}SELECT\PYGZus{}SET}\PYG{p}{,} \PYG{l+m+mi}{4}\PYG{p}{,} \PYG{p}{(}\PYG{n}{const} \PYG{n}{hssize\PYGZus{}t} \PYG{o}{*}\PYG{o}{*}\PYG{p}{)}\PYG{n}{coord}\PYG{p}{)}\PYG{p}{;}
\end{sphinxVerbatim}



\renewcommand{\indexname}{Index}
\printindex
\end{document}